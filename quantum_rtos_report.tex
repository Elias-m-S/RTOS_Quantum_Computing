\documentclass[11pt,a4paper]{article}

% Packages
\usepackage[utf8]{inputenc}
\usepackage[T1]{fontenc}
\usepackage[english]{babel}
\usepackage{amsmath}
\usepackage{graphicx}
\usepackage{float}
\usepackage[margin=2.5cm]{geometry}
\usepackage{hyperref}
\usepackage{cite}
\usepackage{caption}
\usepackage{subcaption}
\usepackage{booktabs}
\usepackage{enumitem}

% Document Information
\title{Quantum Computing and RTOS}
\author{Group: Motorcycle}
\date{\today}

\begin{document}

\maketitle

\section{Introduction to Schedulers in RTOS}
The most important part of a real-time operating system is its scheduler. The purpose of a scheduler is to distribute the resources of the processor among multiple tasks in regard to efficiency and optimum utilization but a scheduler for a real-time system shall also guarantee to meet a specific deadline for each task. To accomplish those requirements several scheduling strategies have been developed.

\section{Introduction to Quantum Computing}
A classic computer calculates with two states known as 0 and 1 and are physically represented by a voltage level. But on a quantum computer the information is represented by qubits which can be created by spinning electrons or atoms. Each qubit has basis states that are distinct or also called orthogonal. That means that if the qubit is in one of the basis states it cannot be in the other basis state. Besides that, qubits can be in superposition. Those superpositions can be illustrated by a spinning coin. When a spinning coin is collapses it will be in a basis state, head or tail, but while spinning the coin is in both of those states simultaneously. This principle practically allows to do calculations with a 0 and 1 at the same time. In order to control the signals, like in a classic computer by NAND or XOR gates, the gates on a quantum computer are implemented by lasers which affect the qubits.

\section{Architecture of quantum computing system}
The essential part of a quantum computer is the QPU (Quantum Processing Unit). However, the classical computing parts such as CPU are used to validate the results of quantum calculations and control the qubits. Therefore, synchronization and coordination of both parts is essential.

\section{Relevance of real-time scheduling in quantum computing}
The state of the physical qubit is unstable and can be easily corrupted by extern factors as thermal noise energy, vibrations and etc. Therefore, the control system that relies on classic computing and enables the reading of the qubit state, error correction and the regulation of the laser at quantum gates must operate in real-time.

\section{Challenges in real-time scheduling for quantum computing systems}

\section{Approaches to real-time scheduling}

% Bibliography
\begin{thebibliography}{99}

\bibitem{ref1}
% Add your references here
% Example: Author Name, "Title of Paper", Journal/Conference, Year.

\bibitem{ref2}
% Reference 2

\bibitem{ref3}
% Reference 3

\bibitem{ref4}
% Reference 4

\bibitem{ref5}
% Reference 5

\end{thebibliography}

\end{document}
