\documentclass[11pt,a4paper]{article}

% Packages
\usepackage[utf8]{inputenc}
\usepackage[T1]{fontenc}
\usepackage[english]{babel}
\usepackage{amsmath}
\usepackage{graphicx}
\usepackage{float}
\usepackage[margin=2.5cm]{geometry}
\usepackage{hyperref}
\usepackage{cite}
\usepackage{caption}
\usepackage{subcaption}
\usepackage{booktabs}
\usepackage{enumitem}

% Document Information
\title{Quantum Computing and RTOS: \\ Exploring Real-Time Scheduling for Quantum Computing Systems}
\author{Group: Motorcycle}
\date{\today}

\begin{document}

\maketitle

\begin{abstract}
% Fill in your abstract here (150-200 words)
% Briefly introduce quantum computing and RTOS integration
% Mention the main challenges and approaches discussed
\end{abstract}

\section{Introduction}
% Fill in your introduction here
% Introduce quantum computing
% Introduce real-time operating systems (RTOS)
% Explain why real-time scheduling is important for quantum computing systems
% Provide overview of the report structure

\section{Background}
\subsection{Quantum Computing Fundamentals}
% Provide brief background on quantum computing concepts
% Qubits, superposition, entanglement
% Quantum gates and circuits

\subsection{Real-Time Operating Systems}
% Brief introduction to RTOS concepts
% Real-time scheduling requirements
% Hard vs. soft real-time systems

\section{Challenges in Real-Time Scheduling for Quantum Computing Systems}
% Main section - discuss various challenges

\subsection{Quantum Coherence Time Constraints}
% Discuss the challenge of decoherence
% Time-sensitive nature of quantum operations
% Impact on scheduling decisions

\begin{figure}[H]
    \centering
    % \includegraphics[width=0.8\textwidth]{figures/coherence_time.png}
    \caption{Example: Quantum coherence time decay and its impact on scheduling windows}
    \label{fig:coherence}
\end{figure}

\subsection{Quantum Error Correction Overhead}
% Discuss error correction requirements
% Trade-offs between error correction and execution time
% Scheduling implications

\subsection{Resource Allocation and Qubit Mapping}
% Physical qubit limitations
% Mapping logical to physical qubits
% Scheduling constraints from hardware topology

\begin{figure}[H]
    \centering
    % \includegraphics[width=0.7\textwidth]{figures/qubit_mapping.png}
    \caption{Example: Qubit connectivity graph and mapping constraints}
    \label{fig:qubit_mapping}
\end{figure}

\subsection{Classical-Quantum Interaction Delays}
% Communication between classical and quantum processors
% Latency considerations
% Hybrid algorithm scheduling challenges

\subsection{Non-Deterministic Execution Times}
% Variability in quantum measurements
% Probabilistic nature of quantum algorithms
% Challenges for traditional RTOS schedulers

\section{Approaches to Real-Time Scheduling}
% Main section - discuss various approaches and solutions

\subsection{Priority-Based Scheduling}
% Adapt traditional priority scheduling for quantum tasks
% Assigning priorities based on coherence time requirements
% Preemption challenges in quantum context

\subsection{Deadline-Driven Scheduling}
% EDF (Earliest Deadline First) adaptations
% Setting deadlines based on decoherence limits
% Feasibility analysis for quantum workloads

\begin{figure}[H]
    \centering
    % \includegraphics[width=0.8\textwidth]{figures/scheduling_approach.png}
    \caption{Example: Comparison of different scheduling approaches for quantum tasks}
    \label{fig:scheduling}
\end{figure}

\subsection{Time-Slicing and Quantum Circuit Partitioning}
% Breaking quantum circuits into schedulable units
% Intermediate measurements and state preservation
% Trade-offs between partitioning and overhead

\subsection{Hybrid Classical-Quantum Scheduling}
% Co-scheduling classical and quantum resources
% Communication-aware scheduling
% Optimization algorithms for hybrid systems

\subsection{Predictive and Adaptive Scheduling}
% Learning-based approaches
% Adapting to hardware characteristics
% Online vs. offline scheduling strategies

\section{Comparison and Analysis}
% Optional: Compare different approaches
% Discuss advantages and disadvantages
% Performance metrics and trade-offs

% Optional: Include a table comparing approaches
\begin{table}[H]
    \centering
    \caption{Comparison of Real-Time Scheduling Approaches for Quantum Systems}
    \label{tab:comparison}
    \begin{tabular}{@{}llll@{}}
        \toprule
        \textbf{Approach} & \textbf{Advantages} & \textbf{Disadvantages} & \textbf{Use Case} \\ 
        \midrule
        Priority-Based & & & \\
        \midrule
        Deadline-Driven & & & \\
        \midrule
        Time-Slicing & & & \\
        \midrule
        Hybrid & & & \\
        \midrule
        Adaptive & & & \\
        \bottomrule
    \end{tabular}
\end{table}

\section{Conclusion}
% Summarize key challenges identified
% Recap main scheduling approaches
% Future directions and open problems
% Final thoughts on the integration of RTOS and quantum computing

\section{Future Work}
% Optional: Discuss future research directions
% Emerging technologies and approaches
% Open challenges

% Bibliography
\begin{thebibliography}{99}

\bibitem{ref1}
% Add your references here
% Example: Author Name, "Title of Paper", Journal/Conference, Year.

\bibitem{ref2}
% Reference 2

\bibitem{ref3}
% Reference 3

\bibitem{ref4}
% Reference 4

\bibitem{ref5}
% Reference 5

\end{thebibliography}

\end{document}
